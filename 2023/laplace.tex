\begin{enumerate}[label=\thechapter.\arabic*,ref=\thechapter.\theenumi]

\item The number of zeroes of the polynomial $P(s) = s^3+2s^2+5s+80$ in the right side of the plane?\hfill(GATE IN 2023) \\

\solution
\input{2023/IN/24/main.tex}
\newpage

\item The circuit shown in the figure is initially in the steady state with the switch K in open condition and $\overline{K}$ in closed condition. The switch K is closed and $\overline{K}$ is opened simultaneously at the instant $t = t_1$, where $t_1 > 0$. The minimum value of $t_1$ in milliseconds such that there is no transient in the voltage across the 100 $\mu F$ capacitor, is \rule{1cm}{0.15mm} (Round off to 2 decimal places) \hfill (GATE EE 2023)
\input{2023/EE/54/figs/ckt1.tex}

\newpage
\item $y=e^{mx}+e^{-mx}$ is the solution of which differential equation?
\begin{enumerate}[label=\textbf{\arabic*.}, font=\bfseries, align=left]
    \item $\frac{dy}{dx} - my = 0$ 
    \item $\frac{dy}{dx} + my = 0$ 
    \item $\frac{d^{2}y}{dx^{2}} + m^{2}y = 0$ 
    \item $\frac{d^{2}y}{dx^{2}} - m^{2}y = 0$ 
\end{enumerate} \hfill(GATE AG 2023)
\solution

\newpage
\item  A cascade control strategy is shown in the figure below. The transfer function between the output $(y)$ and the secondary disturbance $(d_2)$ is defined as  \\
$$G_{d2}(s)= \frac{y(s)}{d_2(s)}$$. 
Which one of the following is the CORRECT expression for the transfer function $G_{d2}(s)$? \\
\begin{figure}[h]
    \centering
    \includegraphics[scale=0.25]{2023/CH/44/figs/g44fig1.jpeg}
    \caption{ }
    \label{}
\end{figure}
\begin{enumerate}[label=\Alph*.]
\item $\frac{1}{(11s+21)(0.1s+1)}$ 
\item $\frac{1}{(s+1)(0.1s+1)}$
\item $\frac{(s+1)}{(s+2)(0.1s+1)}$
\item $\frac{(s+1)}{(s+1)(0.1s+1)}$
\end{enumerate} \hfill (GATE CH 2023)
\solution
\newpage
\item In the differential equation $\frac{dy}{dx} + \alpha x y = 0, \alpha$ is a positive constant. If $y = 1.0$ at
$x = 0.0$, and $y = 0.8$ at $x = 1.0$, the value of $\alpha$ is (rounded off to three decimal places).  \hfill(GATE CE 2023)
\solution

\newpage
\item The switch $S_1$ was closed and $S_2$ was open for a long time. At t=0,switch $S_1$ is opened and $S_2$ is closed,simultaneously. The value of $i_c(0^{+})$, in amperes, is . \hfill (GATE EC 2023)\\
\input{2023/EC/44/figs/ckt1.tex}
\newpage

\item The continuous time signal $x(t)$ is described by:
\begin{align}
x(t)=
    \begin{cases}
        1, & \text{if } 0\: {\displaystyle \leq }\:t\:{\displaystyle \leq }\:1\\
        0, & \text{elsewhere}
    \end{cases} 
\end{align}
If $y(t)$ represents $x(t)$ convolved with itself, which of the following options is/are TRUE?
\begin{enumerate}[label = \Alph*]
    \item $y(t)$ = 0 for all $t<0$\\
    \item $y(t)$ = 0 for all $t>1$\\
    \item $y(t)$ = 0 for all $t>3$\\
    \item $\int_{0.1}^{0.75} \frac{dy(t)}{dt}\: \text{dt} \neq 0$
\end{enumerate}
\solution
\newpage

\item The Z-transform of a discrete signal $x\brak{n}$ is
\begin{align}
X\brak{z}=\dfrac{4z}{\brak{z-\dfrac{1}{5}} \brak{z-\dfrac{2}{3}} \brak{z-3}} \text{ with ROC= }R
\end{align}
Which one of the following statements is TRUE?
\begin{enumerate}[label = (\alph*)]
     \item Discrete time Fourier transform of $x\sbrak{n}$ converges if $R$ is $|z|>3$\\
     \item Discrete time Fourier transform of $x\sbrak{n}$ converges if $ R$ is $\dfrac{2}{3}<|z|<3$\\
     \item Discrete time Fourier transform of $x\sbrak{n}$ converges if $R$ is such that $x\sbrak{n}$ is a left-sided sequence.\\
     \item Discrete time Fourier transform of $x\sbrak{n}$ converges if $R$ is such that $x\sbrak{n}$is a right-sided sequence.\\
 \end{enumerate} \hfill{GATE EE 2023}
 \solution
 \newpage
 
\item The phase margin of the transfer function $G(s) = \frac{2(1-s)}{(1+s)^2}$ is \rule{1cm}{0.15mm} degrees. (rounded off to the nearest integer). \hfill (GATE IN 2023)\\
\solution
\newpage
\item Consider the second-order linear differential equation
\[x^2\frac{d^2y}{dx^2}+x\frac{dy}{dx}-y=0, \; x\geq 1\]
with the initial conditions \[y(x=1)=6,\; \;\; \frac{dy}{dx}\big{|}_{x=1}=2.\]
Then the value of $y$ at $x=2$ is \rule{2cm}{0.1mm}.\\{\hfill{GATE ME 2023}}\\
\solution
\newpage
\item The transfer function of a measuring instrument is \\
$$G_m(s) = \frac{1.05}{2s+1}exp(-s)$$
At time $t = 0$, a step change of +1 unit is introduced in the input of this instrument.The time taken by the instrument to show an increase of 1 unit in its output is(rounded off to two decimal places).\\ \hfill (GATE CH 2023)
\solution
\item
The laplace transform of $x_1(t)$ = $e^{-t}u(t)$ is $X_1(s)$, where $u(t)$ is the unit step function. The laplace transform of $x_2(t) = e^tu(-t)$ is $X_2(s)$. Which one of the following statements is TRUE?
\begin{enumerate}
    \item The region of convergence of $X_1(s)$ is $Re(s) \geq 0$
    \item The region of convergence of $X_2(s)$ is confined to the left half-plane of s.
    \item The region of convergence of $X_1(s)$ is confined to the right half-plane of s.
    \item the imaginary axis in the s-plane is included in both the region of convergence of $X_1(s)$ and the region of convergence of $X_2(s)$.
\end{enumerate} \hfill(GATE BM 2023)\\
\solution
\newpage
\item Given that $\frac{dy}{dx}=2x+y$ and $y=1$,when $x=0$ Using Runge-Kutta fourth order method,the value of $y$ at $x=0.2$ is \hfill(GATE 2023 AG 50) \\
\solution
\newpage
\item $y = ae^{mx} + be^{-mx}$ is the solution of the differential equation
\begin{enumerate}[label=\alph*)]
    \item $\frac{dy}{dx} - my = 0$
    \item $\frac{dy}{dx} + my = 0$
    \item $\frac{d^2y}{dx^2} + m^2y = 0$
    \item $\frac{d^2y}{dx^2} - m^2y = 0$
\end{enumerate}
 \hfill(GATE 2023 AG 14)

 \solution
\end{enumerate}
